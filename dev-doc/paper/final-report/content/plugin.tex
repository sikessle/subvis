\chapter{Subdivision Plugin}

Die Hauptfunktionalität des Programms SubVis wird ebenfalls in einem Plugin bereitgestellt. 
Dieses erlaubt die Anwendung verschiedener Unterteilungsalgorithmen und das Rendering deren Limes-Flächen.

\begin{figure}
  \centering
  \includegraphics[width=\textwidth]{content/media/subvis_architektur_plugin_subdivision.png}
  \caption{Architektur des Subdivision-Plugins}
  \label{fig:subvis_architektur_plugin_subdivision}
\end{figure}

\autoref{fig:subvis_architektur_plugin_subdivision} gibt einen Überblick über die Architektur.
Die Hauptklasse ist \emph{SubdivisionAlgorithmsPlugin} welche auch die SubVisPlugin Schnittstelle implementiert.
Die GUI, das Rendering und die Algorithmen wurden in jeweils eigene Klassen ausgelagert.

\section{GUI}

\begin{figure}
  \centering
  \includegraphics[width=0.3\textwidth]{content/media/subvis_plugin.png}
  \caption{GUI-Elemente des Plugins}
  \label{fig:subivs_plugin_gui}
\end{figure}

Die GUI besteht aus zwei Auswahllisten für den Algorithmus pro Polygonnetz, einer Einstellung für die Anzahl an Unterteilungsschritten und einem Start bzw. Stop Button (siehe \autoref{fig:subivs_plugin_gui}).

\section{Rendering}
Das Rendering der Limesfläche erfolgt mittels dreier unterschiedlicher Renderer.
Zur Verfügung stehen ein interpolierender Renderer, ein B-Spline Renderer und ein Box-Spline Renderer.
Jeder Renderer erweitert die abstrakte Klasse \emph{GLRenderer}. 
Diese ruft mittels Template-Pattern eine Methode \emph{render} ihrer Unterklasse auf, um den Rendervorgang zu starten.
Ziel ist es die Limesfläche des Kontrollnetzes zu rendern.
Abhängig vom angewendeten Unterteilungsalgorithmus wird die Limesfläche unterschiedlich erstellt.

\subsection{Interpolating Renderer}
Für die Unterteilungsalgorithmen Butterfly und Modified Butterfly ist es nicht möglich eine Limesfläche zu berechnen.
Bei diesen Algorithmen handelt es sich um interpolierende Algorithmen, bei denen die Kontrollpunkte Teil der Limesfläche sind.
Daher kommt hierbei ein interpolierender Renderer zum Einsatz.
Der interpolierende Renderer wendet den gewählten Unterteilungsalgorithmus dreifach an, wodurch sich eine für den Benutzer glatte Oberfläche ergibt. 
Zusätzlich wird das OpenGL \textit{Smooth Shading Model} verwendet, wodurch die einzelnen Flächen (Faces) noch weniger gut sichtbar sind und die Oberfläche für den Benutzer noch glatter erscheint.

\subsection{B-Spline Renderer}
Die Algorithmen Catmull-Clark und Doo-Sabin sind Unterteilungsalgorithmen auf Basis von quadratischen und kubischen B-Splines.
Der B-Spline Renderer rendert die Fläche unter Verwendung des GLU NURBS Interfaces.
Hierbei werden die Punkte des Kontrollnetzes als Kontrollpunkte interpretiert und die Limesfläche evaluiert und gerendert.
Die Auswertung erfolgt hierbei schrittweise.
Es ist nicht möglich, da vollständige Kontrollnetz in einem Schritt auszuwerten.
Daher werden jeweils Teilnetze ausgewertet und gerendert.
Dazu wird über alle Flächen (Faces) des Kontrollnetzes iteriert.
Dabei werden die angrenzenden Flächen zu jeder Fläche bestimmt und über deren Ecken wiederum iteriert.
Daraus ergeben sich letztendlich die Kontrollpunkte der Teilnetze.
Da es sich um B-Splines handelt, stellt die Überlappung der Teilnetze kein Problem dar.
Dieser Renderer ist in der aktuellen Version durch das \textit{Mock Rendering} ersetzt, wird aber in der nächsten Version korrekt umgesetzt.

\subsection{Box-Spline Renderer}
Das Rendering der Limesfläche des Loop Unterteilungsalgorithmus erfolgt mit einem interpolierenden Renderer.
Dazu wird der Loop Unterteilungsalgorithmus dreifach angewendet und die so entstandene Fläche analog zum \textit{Interpolating Renderer} gerendert.
Da der Loop Unterteilungsalgorithmus auf Box-Splines basiert, kann eine Bezier Repräsentation des Kontrollnetzes berechnet werden \cite{prautzsch:2002} und das so entstandene Kontrollnetz mit OpenGL Evaluators evaluiert und gerendert werden.
Alternativ dazu kann die Limesfläche auch mit der von Jos Stam beschriebenen Methode \cite{JosStam.24.07.2015} evaluiert und gerendert werden.

\subsection{None Renderer}
Der None Renderer ist eine Dummy Implementierung für den Fall, dass kein Unterteilungsalgorithmus ausgewählt ist.
Dieser ist lediglich auf Grund der Implementierung von SubVis notwendig. In diesem Fall wird kein Rendering ausgeführt.

\subsection{Optimierung}
OpenGL bietet Optionen, um das Rendering zu optimieren.
Beim Rendern des Netzes werden die Koordinaten der Ecken (Vertices) aller Flächen (Faces) an die Grafikkarte übertragen.
Da die meisten Ecken Teil mehrerer Flächen sind, kommt es zu mehrfacher Übertragung von Koordinaten.
Um die Menge an übertragenen Daten zu minimieren, besteht die Möglichkeit Ecken (Vertices) getrennt von den zugehörigen Koordinaten zu übertragen.
Hierbei werden die Koordinaten in einer mathematischen Menge gespeichert und den Ecken (Vertices) werden jeweils Indizes zu den zugehörigen Koordinaten in der Menge zugeordnet.
So kann sichergestellt werden, dass keine Duplikate von Koordinaten übertragen werden.
Diese Optimierung wird als \glqq Indizierung\grqq bezeichnet.

Eine weitere Option zur Optimierung ist das \glqq Depth Testing\grqq.
Es besteht darin, das Rendering von Ecken (Vertices), Kanten (Edges) und Flächen (Faces) nur auf die für die Kamera sichtbaren Bereiche zu beschränken.
Hierzu werden die Entfernungen von Elementen zur Kamera vor dem Rendering verglichen.
Nur die Elemente, deren Entfernung zur Kamera am geringsten sind, werden tatsächlich gerendert.

\subsection{Mock Rendering}
Dieses Rendering dient als temporäre Umsetzung des Renderings der Limesflächen der Unterteilungsalgorithmen Catmull-Clark und Doo-Sabin.
Der Renderer basiert auf dem \enquote{Interpolating Renderer} und wird mit der nächsten Version von SubVis durch den B-Spline Renderer ersetzt.

\section{Algorithmen}

Die Algorithmen werden über die abstrakte Klasse \emph{SdAlgorithm} zusammengefasst. 
Diese ermöglicht es der GUI, abzufragen, ob ein bestimmtes Polygonnetz von diesem Algorithmus unterteilt werden kann oder nicht, was sich in der Auswahlmöglichkeit der GUI niederschlägt.
Außerdem implementiert die Klasse die Verwaltung der Polygonnetze und ein Thread-basiertes unterteilen. 
Somit müssen die konkreten Klassen der Algorithmen lediglich eine Methode überschreiben und sich nur auf die korrekte Implementierung fokussieren.
Es werden vor jedem Aufruf der Unterklasse die Variablen \emph{input\_mesh} und \emph{output\_mesh} entsprechend initialisiert.
Die Unterklassen müssen dann pro Aufruf von \emph{subdivide\_input\_mesh\_write\_output\_mesh} das Eingangsnetz lesen und das Ergebnis in der Variable output\_mesh speichern.

\begin{lstlisting}[style=myCppStyle, caption={Signatur der Unterteilungsfunktion}, label=lst:subdiv_threaded]
void subdivide_threaded(const Surface_mesh& mesh, 
std::function<void(std::unique_ptr<Surface_mesh>)> callback, 
const int steps = 1);
\end{lstlisting}

Das Thread-basierte Rendering ist über einen Callback-Mechanismus implementiert (vgl. \autoref{lst:subdiv_threaded}), welcher garantiert, dass die Callback-Funktion auf dem UI-Thread ausgeführt wird.
Die Funktion subdivide\_threaded ruft eine Worker-Funktion mittels QtConcurrent::run auf, welche in einer Schleife die konkrete Implementierung der Unterteilung ausführt.
Die Worker-Funktion sendet am Ende ihrer Ausführung ein Signal \emph{finished} aus.
Dieses wiederum wird von der Klasse empfangen und in einen Aufruf der Callback-Funktion umgesetzt.
Durch Verwendung des Signal-Slot Konzepts ist garantiert, dass der Slot und somit der Callback auf dem UI-Thread ausgeführt wird.
Der implementierte Callback-Mechanismus macht somit das Threading wesentlich einfacher und benötigt keine Synchronisierungsmechanismen.

Als konkreter Callback wird eine Funktion verwendet, die bei Beendigung das Ergebnis in die Model-Komponente lädt (was ein erneutes Zeichnen des Polygonnetzes zur Folge hat).

\subsection{Vererbungshierarchie}

\emph{SdAlgorithm} ist die Klasse von der alle Unterteilungsalgorithmen abgeleitet werden.
Die Klasse implementiert wie bereits beschrieben alle Basisfunktionen.
Die Algorithmen Catmull-Clark, Doo-Sabin, Loop, Butterfly und Modified Butterfly lassen sich aber
noch weiter klassifizieren und zusammenfassen.
Daher werden die beiden Unterklassen \emph{SdQuad} und \emph{SdTriangle} eingeführt,
die von \emph{SdAlgorithm} erben.

\emph{SdQuad} vereinigt gemeinsame Funktionen von Catmull-Clark und Doo-Sabin.
Dazu gehört das Verwalten von Properties (Edge Point, Face Point \ldots) und
die Berechnung des Face Points.

\emph{SdTriangle} fasst die Unterteilungsalgorithmen für Dreiecksnetze zusammen.
Loop, Butterfly und Modified Butterfly verwenden die gleiche Face Split Funktion
und benötigen ähnlichen Properties.

Die konkreten Implementierungen der Unterteilungsalgorithmen erben schließlich
von \emph{SdQuad} oder \emph{SdTriangle}.
Durch diese Architektur wird redundanter Code vermieden.


\subsection{Catmull-Clark}

Die Implementierung von Catmull-Clark ist wie in \autoref{subsec:catmull}
beschrieben durchgeführt. \emph{SdCatmull} erbt von \emph{SdQuad}.
Der Algorithmus kann mit einem beliebigen Polygonnetz umgehen.
Die Randregeln sind vollständig implementiert.
\autoref{fig:sd_catmull_screenshot} zeigt einen Würfel und das Netz nach
einer Unterteilung.

\begin{figure}
  \centering
  \includegraphics[width=0.7\textwidth]{content/media/sd_catmull_screenshot.png}
  \caption{SubVis - Catmull-Clark}
  \label{fig:sd_catmull_screenshot}
\end{figure}


\subsection{Doo-Sabin}

Der Doo-Sabin Algorithmus ist inklusive Randregel wie in \autoref{subsec:doosabin}
beschrieben umgesetzt. Die Klasse \emph{SdDooSabin} erbt von \emph{SdQuad}.
\autoref{fig:sd_catmull_screenshot} zeigt einen Unterteilungsschritt eines Würfels.

\begin{figure}
  \centering
  \includegraphics[width=0.7\textwidth]{content/media/sd_doosabin_screenshot.png}
  \caption{SubVis - Doo-Sabin}
  \label{fig:sd_doosabin_screenshot}
\end{figure}

\subsection{Loop}

Loop ist in der Klasse \emph{SdLoop} implementiert und erbt von \emph{SdTrianlge}.
Die Implementierung erfolgt wie in \autoref{subsec:loop} beschrieben
und unterstützt die beschriebenen Randregeln.
Die Berechnung der Even Vertices erfolgt dabei nach der Variation nach Warren.
\autoref{fig:sd_loop_screenshot} zeigt die Unterteilung eines Icosahedrons.

\begin{figure}
  \centering
  \includegraphics[width=0.7\textwidth]{content/media/sd_loop_screenshot.png}
  \caption{SubVis - Loop}
  \label{fig:sd_loop_screenshot}
\end{figure}

\subsection{Butterfly}

Der Butterfly Algorithmus ist nach \autoref{subsec:butterfly}
inklusive Randregel in der Klasse \emph{SdButterfly} implementiert.
Für die Unterteilung wird das klassische Eight-Point Stencil verwendet.
\emph{SdButterfly} erbt von \emph{SdTrianlge}.
\autoref{fig:sd_butterfly_screenshot} zeigt einen Unterteilungsschritt.

\begin{figure}
  \centering
  \includegraphics[width=0.7\textwidth]{content/media/sd_butterfly_screenshot.png}
  \caption{SubVis - Butterfly}
  \label{fig:sd_butterfly_screenshot}
\end{figure}

\subsection{Modified Butterfly}

Der Modified Butterfly Algorithmus ist nach \autoref{subsec:modbutterfly} implementiert.
Die Klasse \emph{SdButterfly} erbt jedoch nicht von \emph{SdTriangle},
sondern von \emph{SdButterfly}, da die Unterteilung
beim Modified Butterfly im regulären Fall bis auf Ausnahmeregeln
identisch mit dem Butterfly Algorithmus ist. In \emph{SdButterfly}
muss für eine korrekte Implementierung somit nur noch die Methode
\emph{compute\_edge\_point} überschrieben werden, die
die Sonderfälle behandelt.
Die Implementierung verwendet das klassische Eight-Point Stencil
(bzw. Ten-Point Stencil mit \(w = 0\)).

Der Modified Butterfly Algorithmus ist mit der Standard Randregel
implementiert.
Die erweiterten Randregeln, die am Ende von \autoref{subsec:modbutterfly}
erwähnt werden, sind nicht umgesetzt.
\autoref{fig:sd_modbutterfly_screenshot} zeigt einen Unterteilungsschritt.

\begin{figure}
  \centering
  \includegraphics[width=0.7\textwidth]{content/media/sd_modbutterfly_screenshot.png}
  \caption{SubVis - Modified Butterfly}
  \label{fig:sd_modbutterfly_screenshot}
\end{figure}














