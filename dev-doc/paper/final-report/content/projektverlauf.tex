\chapter{Projektverlauf}

In den folgenden Abschnitten wird der Projektverlauf beschrieben und 
dabei auf die erreichten und geplanten Ergebnisse eingegangen, sowie der bisherige Ablauf des Projekts bewertet. Außerdem wird auf die Arbeitsteilung eingegangen.

\section{Aufgabenverteilung}

Für die Entwicklung werden die Themen und Aufgaben unter den Teammitgliedern in drei Arbeitspakete aufgeteilt.

\begin{description}
\item[Architektur, Oberfläche, IDE, Editiermodus] Hierzu gehört die Oberfläche, Bedienung, Architektur inkl. Plugin-System und die Spezifikation einer Entwicklungsumgebung. Des Weiteren die Implementierung des Editiermodus.
\item[Rendering und Darstellung] Berechnung und Darstellung der Kontrollnetze und deren Limesfläche mit OpenGL.
\item[Unterteilungsalgorithmen] Umfasst die Implementierung der ausgewählten Unterteilungsalgorithmen.
\end{description}

\begin{table}[h]
\center
\caption{Aufgabenverteilung unter den Teammitgliedern}
\begin{tabular}{c|c}
Arbeitspaket & Teammitglied\\
\hline
Architektur, Oberfläche, IDE, Editiermodus & Simon Kessler \\
Rendering und Darstellung & Tobias Keh \\
Unterteilungsalgorithmen & Felix Born \\
\end{tabular}
\label{tab:aufgabenverteilung}
\end{table}

\autoref{tab:aufgabenverteilung} zeigt die vorgenommene Aufgabenverteilung des letzten Semesters.

\section{Verlauf der beiden Semester}

Im ersten Semester lag der Schwerpunkt darin, einen Überblick über die theoretischen Grundlagen zu erhalten. 
Hierzu gehören die Unterteilungsalgorithmen und die Darstellung der Kontrollnetze.
Ein weiter wichtiger Punkt war die Evaluierung von Bibliotheken, die bereits Algorithmen und vor allem Datenstrukturen implementieren. 
Als Ergebnis wurde die Datenstruktur Surface\_mesh als am geeignetsten bewertet.

Des Weiteren wurde die Architektur entworfen und in einem UML-Diagramm festgehalten.
Außerdem entstand eine einheitliche Entwicklungsumgebung basierend auf Qt und dem Qt Creator unter Ubuntu.
Diese enthält die nötigen Build-Dateien, Verzeichnisse, Bibliotheken und Git-Konfigurationen.
Jedes Teammitglied besitzt somit eine einheitliche, funktionierende Entwicklungsumgebung.
Zusätzlich wurde eine Dokumentationslösung auf Quelltextebene mittels Doxygen implementiert.

Da für alle Beteiligten der Umgang mit C++ und insbesondere Computergrafik ein neues Themengebiet darstellt, war ein ausführlicher theoretischer Einstieg in das Thema notwendig. 
Hierbei hat sich die Unterstützung durch Herrn Prof. Dr. Georg Umlauf und Pascal Laube als sehr hilfreich erweisen.
In den Projektbesprechungen wurden technische Details geklärt und der theoretische Hintergrund erläutert.
Erst durch Verstehen der Details bezüglich der Unterteilungsalgorithmen, des Renderings und der Anwendbarkeit der Unterteilungsalgorithmen auf verschiedene Kontrollnetze konnten die Tools und Bibliotheken entsprechend evaluiert werden.

Während des Semesters wurden lediglich für sehr kurze Zeiträume und eng eingegrenzte Themengebiete die Aufgaben zwischen den Teammitgliedern verteilt.
Dadurch wurde sichergestellt, dass eine gemeinsame Wissensbasis entsteht, um späteren Missverständnissen vorzubeugen.
Dies hat sich als hilfreich erwiesen und ermöglicht es nun an verschiedenen Teilmodulen der Anwendung parallel zu arbeiten.

Im zweiten Semester wurde das Programm SubVis implementiert. 
Dabei wurden kleinere Änderungen an der Architektur gegenüber der Spezifizierung im ersten Semester vorgenommen.
Zuerst wurde eine GUI und Model-Schicht entwickelt. 
Diese dienten als Plattform, um die weiteren Komponenten zu implementieren.
Danach wurden die einzelnen Algorithmen, Renderingverfahren und der Editiermodus ergänzt.
In Anlehnung an agile Entwicklungsmodelle wurden zweiwöchige Meetings mit den Betreuern Prof. Dr. Georg Umlauf und Pascal Laube durchgeführt. 
Dort wurde auch der aktuelle Stand der Software präsentiert oder Fragen geklärt.
So konnte frühzeitig auf Änderungs- und Funktionswünsche eingegangen werden.

Schlussendlich kann das Projekt als erfolgreich bezeichnet werden, da alle vorgesehenen Funktionen und darüber hinaus noch weitere implementiert wurden.
Für alle beteiligten Studenten war dies ein überaus lehrreiches Projekt mit einigen neuen Programmiersprachen und Anwendungsdomänen.





