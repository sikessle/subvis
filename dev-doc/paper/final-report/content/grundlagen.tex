\chapter{Grundlagen}

In diesem Kapitel wird eine kurze Einführung in die Grundlagen von Unterteilungsalgorithmen gegeben.

\section{Unterteilungsalgorithmen}

Unterteilungsalgorithmen erzeugen aus einem Ausgangspolygonnetz eine glatte Fläche.
Die glatte Zielfläche ist dabei der Grenzwert eines unendlichen, rekursiven Verfeinerungsschemas.
\autoref{fig:sd} visualisiert die Anwendung eines Unterteilungsalgorithmus auf eine Kurve und auf eine Fläche.
Nach mehrfacher Anwendung der Unterteilung konvergiert die Kurve oder Fläche gegen die glatte Zielkurve bzw. Zielfläche.

\begin{figure}
  \centering
  \includegraphics[width=0.8\textwidth]{content/media/sd.png}
  \caption{Unterteilungsalgorithmus - Kurve und Fläche \cite{Standford.24.07.2015}}
  \label{fig:sd}
\end{figure}

Unterteilungsalgorithmen kann man anhand ihrer Eigenschaften kategorisieren.
Ein Unterscheidungskriterium betrifft die Art und Weiße, wie unterteilt wird.
Man unterscheidet dabei zwischen \emph{primal} und \emph{dual}.

\begin{description}
 \item[Primal] Bei dieser Strategie wird die Oberfläche unterteilt (\enquote{face split}).
\autoref{fig:sd_primal} stellt diese Methode für ein Dreicksnetz und ein Vierecksnetz dar.
 \item[Dual] Auf der anderen Seite ist es möglich Eckpunkte in mehrere Eckpunkte aufzusplitten (\enquote{vertex split}).
 Diese Methode ist in \autoref{fig:sd_dual} abgebildet.

\end{description}
\begin{figure}
  \centering
  \includegraphics[width=0.7\textwidth]{content/media/sd_primal}
  \caption{Primal (face split) \cite{Standford.24.07.2015}}
  \label{fig:sd_primal}
\end{figure}

\begin{figure}
  \centering
  \includegraphics[width=0.7\textwidth]{content/media/sd_dual}
  \caption{Dual (vertex split) \cite{Standford.24.07.2015}}
  \label{fig:sd_dual}
\end{figure}

Ein weiteres wesentliches Merkmal ist, ob Kontrollpunkte interpoliert werden oder nicht. 
\begin{description}
 \item[Approximation] Kontrollpunkte werden nicht interpoliert.
 \item[Interpolation] Kontrollpunkte werden interpoliert.
\end{description}

\autoref{tab:sd_comp_primal} und \autoref{tab:sd_comp_dual} listet die bekanntesten Unterteilungsalgorithmen auf und ordnet diese den Kategorien zu.
Zu jedem Algorithmus ist zusätzlich die \enquote{Glattheit} der Oberfläche angegeben (C-Stetigkeit).
Die Unterteilungsalgorithmen erzeugen auf beliebig angeordnete Netze \(C^1\) stetige Flächen.
In den \autoref{tab:sd_comp_primal} und \autoref{tab:sd_comp_dual} wird die Stetigkeit für den
\emph{regulären} Fall angegeben.
Die C-Stetigkeit kann auch als Maß über die Qualität des Unterteilungsalgorithmus fungieren.
\begin{table}
\center
\caption{Unterteilungsalgorithmen Übersicht primal \cite[S. 65]{Zorin.subdivcourse}}
\label{tab:sd_comp_primal}
\begin{tabular}{l|c|c|}
& \multicolumn{2}{c|}{\textbf{Primal}}\\
\hline
& \textbf{Dreiecksnetz} & \textbf{Vierecksnetz}\\
\hline
\textbf{Approximation} & Loop \((C^2)\) & Catmull-Clark \((C^2)\)\\
\textbf{Interpolation} & Butterfly \((C^1)\) & Kobbelt \((C^1)\)\\
\end{tabular}
\end{table}
\begin{table}
\center
\caption{Unterteilungsalgorithmen Übersicht dual \cite[S. 65]{Zorin.subdivcourse}}
\label{tab:sd_comp_dual}
\begin{tabular}{c}
\\
\hline
\textbf{Dual}\\
\hline
Doo-Sabin \((C^1)\) \\
Biquartic \((C^2)\) \\
\end{tabular}
\end{table}
\autoref{fig:sd_comp} vergleicht die vier unterschiedlichen Unterteilungsalgorithmen Catmull-Clark, Loop, Doo-Sabin und Butterfly.
Man erkennt deutlich den interpolierenden Unterteilungsalgorithmus (Butterfly),
da dieser durch die harten Interpolationsbedingungen im Vergleich zu den approximierenden Algorithmen viel \enquote{welliger} ist. \cite{Zorin.subdivcourse}
\begin{figure}
  \centering
  \includegraphics[width=0.9\textwidth]{content/media/sd_overview.png}
  \caption{Vergleich der Unterteilungsalgorithmen \cite{Standford.24.07.2015}}
  \label{fig:sd_comp}
\end{figure}

\section{Auswahl von Unterteilungsalgorithmen}

Für das Projekt sollen folgende Algorithmen implementiert werden:
\begin{itemize}
	\item Catmull-Clark
	\item Loop
	\item Doo-Sabin
	\item Butterfly
\end{itemize}
Dies sind die wichtigsten Vertreter für Dreiecks- und Vierecksnetze.
Prinzipiell sind jedoch noch weitere Algorithmen denkbar.
\cite[S. 65ff]{Zorin.subdivcourse} \cite{Standford.24.07.2015}

\section{Catmull-Clark}

\subsection{Allgemein}

Der Catmull-Clark Algorithmus wurde 1978 von Edwin Catmull und James Clark entwickelt.
Er ist eine Verallgemeinerung von bi-cubic uniform B-splines surfaces und arbeitet auf
Netzen mit beliebiger Topologie.
Das neu erzeugte Netz ist immer ein Vierecksnetz. Jedes n-Gon im Eingabenetz wird in n Quads
im Ausgabenetz umgewandelt.
Die Kontrollpunkte des Netzes werden durch Unterteilung approximiert.


\subsection{Unterteilungsregeln}

\begin{figure}
\centering
\includegraphics[width=0.6\textwidth]{content/media/sd_catmull_mask.png}
\caption{Catmull-Clark Maske \cite{yoshihitoyagi.23.12.2015}}
\label{fig:sd_catmull_mask}
\end{figure}

\cite{rosettacode.23.12.2015}
\cite{rorydriscoll.23.12.2015}
\cite{yoshihitoyagi.23.12.2015}



\subsection{Sonderfälle}



\section{Loop}

\subsection{Allgemein}

Charles Loop hat 1987 einen Unterteilungsalgorithmus für Dreiecksnetze entwickelt.
Der Loop Algorithmus basiert auf quartischen Box Splines und approximiert die Kontrollpunkte.
An extraordinären Stellen mit Valence ungleich sechs erzeugt Loop \(C^1\) stetige Flächen,
im regulären Fall ansonsten \(C^2\).
\cite[S. 70f]{Zorin.subdivcourse} \cite[S. 56f]{Standford.24.07.2015}

\subsection{Unterteilungsregel}

Die Unterteilung erfolgt in drei Schritten.
\begin{enumerate}
\item Berechne für jede Kante einen edge point. Dieser wird auch als odd Vertex bezeichnet.
\item Berechne für jeden Vertex eine neue Position. Dieser wird auch als even Vertex bezeichnet.
\item Ersetze jede Dreieck durch vier neue Dreiecke.
\end{enumerate}

\begin{figure}
\centering
\includegraphics[width=1.0\textwidth]{content/media/sd_loop_mask.jpg}
\caption{Loop Maske \cite[S. 70f]{Zorin.subdivcourse}}
\label{fig:sd_loop_mask}
\end{figure}

\subsubsection*{Odd Vertex}

\subsubsection*{Even Vertex}

\subsubsection*{Face Split}


\begin{figure}
\centering
\includegraphics[width=0.3\textwidth]{content/media/sd_loop_split.png}
\caption{Loop face split \cite[S. 56f]{Standford.24.07.2015}}
\label{fig:sd_loop_split}
\end{figure}

\subsection{Boundary}


\section{Doo-Sabin}

\subsection{Allgemein}

Der Algorithmus Doo-Sabin wurde 1978 von Daniel Doo and Malcolm Sabin entwickelt.
Doo-Sabin ist eine Verallgemeinerung von bi-quadratischen uniformen B-Spline Flächen
und kann auf Netzen mit beliebigen Polygonen arbeiten.
Das verfeinerte Ergebnis nach einem Unterteilungsschritt besteht hauptsächlich
aus Vierecken, an extraordinären Stellen jedoch auch aus beliebigen Polygonen.
Die Kontrollpunkte werden approximiert.
Doo-Sabin erzeugt im Regelfall \(C^1\) stetige Fläche.
An bestimmten extraordinären Stellen liefert der Algorithmus allerdings lediglich \(C^0\)
(in der Mitte eines irregulären Polygons nach dem ersten Unterteilungschritt).
\cite[S. 60]{Standford.24.07.2015} \cite[S. 79f]{Zorin.subdivcourse}

\subsection{Unterteilungsregel}



\subsection{Randregel}



\section{Butterfly} \label{sec:butterfly}

\subsection{Allgemein}

Der Butterfly Algorithmus ist ein interpolierender Unterteilungsalgorithmus,
der von Nira Dyn, David Levine und John A. Gregory entwickelt wurde.
Butterfly arbeitet auf Dreiecksnetzen und erzeugt an regulären Stellen
\(C^1\) stetige Flächen, an extraordinären Stellen
(Valenz gleich drei oder größer sieben) jedoch lediglich \(C^0\).
\cite[S. 64ff]{Standford.24.07.2015} \cite[S. 72ff]{Zorin.subdivcourse}
\cite{Seeger01asub-atomic}
\cite{Gamasutra}
\cite{Sharp}
\cite{Zorin:1996:ISM:237170.237254}

\subsection{Unterteilungs- und Randregeln}

Der Algorithmus besteht aus zwei Schritten:
\begin{enumerate}
\item Berechne für jede Kante einen Edge Point.
\item Ersetze jedes Dreieck durch vier neue Dreiecke.
\end{enumerate}

\begin{figure}
\centering
\includegraphics[width=0.8\textwidth]{content/media/sd_butterfly_mask.jpg}
\caption{Butterfly Eight-Point Stencil und Randregel \cite{Seeger01asub-atomic}}
\label{fig:sd_butterfly_mask}
\end{figure}

\subsubsection*{Edge Point}
Die Berechnung des Edge Points wird mit dem sogenannten Eight-Point Stencil
aus \autoref{fig:sd_butterfly_mask} durchgeführt.
Die Regel für den Randfall ist dort auch dargestellt.


Da die Punkte beim Butterfly interpoliert werden, müssen die alten Vertices
(wie bisher bei approximierenden Algorithmen) nicht neu berechnet werden. 

\subsubsection*{Face Split}
Der Face Split ist identisch wie bei Loop.
Jedes Dreieck wird in vier neue Dreiecke aufgeteilt (\autoref{fig:sd_loop_split}).
\cite[S. 64ff]{Standford.24.07.2015} \cite[S. 72ff]{Zorin.subdivcourse}
\cite{Seeger01asub-atomic}
\cite{Gamasutra}
\cite{Sharp}
\cite{Zorin:1996:ISM:237170.237254}
\section{Modified Butterfly} \label{sec:modbutterfly}

\subsection{Allgemein}

Der Modified Butterfly ist eine Erweiterung des klassichen
Butterfly Algorithmus und wurde von Denis Zorin, Peter Schröder und Wim Sweldens entwickelt.
Der Unterteilungsalgorithmus garantiert mit der Modifikation \(C^1\) stetige Flächen
für beliebige Dreiecksnetze.
\cite[S. 72ff]{Zorin.subdivcourse}
\cite{Gamasutra}
\cite{Sharp}

\subsection{Unterteilungs- und Randregeln}

Der Modified Butterfly besteht wie der Butterfly aus nur zwei Schritten:
\begin{enumerate}
\item Berechne für jede Kante einen Edge Point.
\item Ersetze jedes Dreieck durch vier neue Dreiecke.
\end{enumerate}
Der Unterschied liegt darin, dass die Edge Points beim Modified Butterfly
unterschiedlich berechnet wird. Die Berechnung hängt davon ob,
ob die andliegenden Vertices der Kante regulär oder extraordinär sind.

\subsubsection*{Edge Point}

\begin{figure}
\centering
\includegraphics[width=0.8\textwidth]{content/media/sd_modbutterfly_mask.jpg}
\caption{Modified Butterfly Ten-Point Stencil und Maske für einen extraordinären Vertex
\cite{Zorin:1996:ISM:237170.237254}}
\label{fig:sd_modbutterfly_mask}
\end{figure}

Für die Berechnung des Edge Points werden vier Fälle unterschieden:
\begin{description}
\item[Kante verbindet zwei reguläre Vertices \((Valenz = 6)\):]
In diesem Fall wird eine Erweiterung
des Butterfly Schemas verwendet. Das sogenannte Ten-Point Stencil ist in \autoref{fig:sd_modbutterfly_mask} links abgebildet.\\
Die Gewichte sind:
\(a = 1/2 - w,\ b = 1/8 + 2w,\ c = -1/16 - w,\ d = w\).\\
\(w\) kann dabei geeignet klein gewählt werden.
Zorin verwendet in \cite{Zorin:1996:ISM:237170.237254} \(w = 0\).
In diesem Fall ist das Ten-Point Stencil identisch mit dem original
Butterfly Eight-Point Stencil.
\item[Kante verbindet K-Vertex \((Valenz \neq 6)\) und Sechs-Vertex \((Valenz = 6)\):]
In diesem Fall wird das rechte Schema aus \autoref{fig:sd_modbutterfly_mask} verwendet.
Abhängig von der Valenz \(K\) werden die Gewichte verschieden gewählt.\\
\(j = 0, \ldots, K - 1\ und\ q = 3/4\)
\begin{itemize}
 \item \(K = 3\): \(s_0 = 5/12,\ s_{1,2} = -1/12\)
 \item \(K = 4\): \(s_0 = 3/8,\ s_{2} = -1/8,\ s_{1,3} = 0\)
 \item \(K >= 5\): \(s_j = (\frac{1}{4}+ cos(\frac{2 \pi j}{K}) + \frac{1}{2} * cos(\frac{4 \pi j}{K}))/K\)
\end{itemize}
\item[Kante verbindet zwei extraordinäre Vertices:]
Es wird nach dem obigen Schema jeweils für beide Vertices ein Edge Point bestimmt und
davon der Durchschnitt errechnet.
\item[Randkante:] Die Randregel ist identisch zur Butterfly Randregel
(\autoref{fig:sd_butterfly_mask}) mit den Gewichten
\(s_{-1} = -1/16,\ s_0 = 9/16,\ s_1 = 9/16,\ s_2 = -1/16\).
\end{description}
\cite{Zorin:1996:ISM:237170.237254}
\cite[S. 72ff]{Zorin.subdivcourse}
\cite{Gamasutra}
\cite{Sharp}


\subsubsection*{Face Split}
Der Face Split ist identisch wie beim Butterfly Algorithmus.
\cite{Zorin:1996:ISM:237170.237254}
\cite[S. 72ff]{Zorin.subdivcourse}

\subsubsection*{Erweiterte Randregeln}

Für den Modified Butterfly Algorithmus gibt es noch eine ganze Reihe von
Randregeln. Diese treten dann ein, wenn die Maske für die Unterteilung zu groß ist
(ein einzelnes Randdreieck \ldots).
Diese Regeln und Sonderfälle werden jedoch ziemlich komplex und aufwändig in der Implementierung
und sind nicht in Subvis implementiert. Daher werden diese hier nicht weiter diskutiert.
Bei weiterem Interesse ist das Dokument \cite[S. 74f]{Zorin.subdivcourse}
empfehlenswert.


\input{content/grundlagen_rendering}
