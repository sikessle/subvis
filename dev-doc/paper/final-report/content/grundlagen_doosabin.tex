\section{Doo-Sabin}

\subsection{Allgemein}

Der Algorithmus Doo-Sabin wurde 1978 von Daniel Doo and Malcolm Sabin entwickelt.
Doo-Sabin ist eine Verallgemeinerung von bi-quadratischen uniformen B-Spline Flächen
und kann auf Netzen mit beliebigen Polygonen arbeiten.
Das verfeinerte Ergebnis nach einem Unterteilungsschritt besteht hauptsächlich
aus Vierecken, an extraordinären Stellen jedoch auch aus beliebigen Polygonen.
Die Kontrollpunkte werden approximiert.
Doo-Sabin erzeugt im Regelfall \(C^1\) stetige Fläche.
An bestimmten extraordinären Stellen liefert der Algorithmus allerdings lediglich \(C^0\)
(in der Mitte eines irregulären Polygons nach dem ersten Unterteilungschritt).
\cite[S. 60]{Standford.24.07.2015} \cite[S. 79f]{Zorin.subdivcourse}

\subsection{Unterteilungsregel}



\subsection{Randregel}


