\chapter{Projektverlauf}

\section{Bisherige Ergebnisse}

Im ersten Semester lag der Schwerpunkt darin, einen Überblick über die theoretischen Grundlagen zu erhalten. 
Hierzu gehören die Unterteilungsalgorithmen und die Darstellung der Kontrollnetze.
Ein weiter wichtiger Punkt war die Evaluierung von Bibliotheken, die bereits Algorithmen und vor allem Datenstrukturen implementieren. 
Als Ergebnis wurde die Datenstruktur Surface\_mesh als am geeignetsten bewertet.

Des Weiteren wurde die Architektur entworfen und in einem UML-Diagramm festgehalten.
Außerdem entstand eine einheitliche Entwicklungsumgebung basierend auf Qt und dem Qt Creator unter Ubuntu.
Diese enthält die nötigen Build-Dateien, Verzeichnisse, Bibliotheken und Git-Konfigurationen.
Jedes Teammitglied besitzt somit eine einheitliche, funktionierende Entwicklungsumgebung.
Zusätzlich wurde eine Dokumentationslösung auf Quelltextebene mittels Doxygen implementiert.

\section{Planung für kommendes Semester}

Im kommenden Semester liegt der Fokus auf der Implementierung der einzelnen Komponenten des Programms.
Das Ergebnis umfasst das spezifizierte Programm \emph{SubVis} und eine schriftliche Ausarbeitung, die den Aufbau und die Umsetzung beschreibt und begründet.
Diese enthält zusätzlich eine Dokumentation die eine Top-Down-Ansicht liefert. 
Implementierungsdetails werden im Quellcode mittels Doxygen dokumentiert.

\subsection{Aufgabenverteilung}

\begin{table}
\center
\caption{Aufgabenverteilung unter den Teammitgliedern}
\begin{tabular}{c|c}
Arbeitspaket & Teammitglied\\
\hline
Architektur, Oberfläche, Entwicklungsumgebung & Simon Kessler \\
Rendering und Darstellung & Tobias Keh \\
Unterteilungsalgorithmen & Felix Born \\
\end{tabular}
\label{tab:aufgabenverteilung}
\end{table}

Für die Entwicklung werden die Themen und Aufgaben unter den Teammitgliedern in drei Arbeitspakete aufgeteilt.

\begin{description}
\item[Architektur, Oberfläche, Entwicklungsumgebung] Hierzu gehört die Oberfläche, Bedienung, Architektur inkl. Plugin-System und die Spezifikation einer Entwicklungsumgebung.
\item[Rendering und Darstellung] In diesem Arbeitspaket soll die Berechnung und Darstellung der Kontrollnetze und deren Limesfläche implementiert werden.
\item[Unterteilungsalgorithmen] Umfasst die Implementierung der vorgegebenen Unterteilungsalgorithmen.
\end{description}

\autoref{tab:aufgabenverteilung} zeigt die geplante Aufgabenverteilung für kommendes Semester.
Die Zuteilung der Arbeitspakete stellt lediglich einen Startzustand dar. 
Um jedem Teammitglied Einblicke in alle Aspekte zu ermöglichen und so auch den größten Lerneffekt zu erzielen, werden Teilaufgaben der Arbeitspakete ausgetauscht.
Der unterschiedliche Arbeitsaufwand wird kompensiert, in dem nach Fertigstellung eines Arbeitspaketes die entsprechende Person bei den anderen Arbeitspaketen weiter entwickelt.

\subsection{Zeitplan}

In den verbleibenden Semesterferien sollen die folgenden Arbeitsschritte durchgeführt werden.
Dabei wird nicht eine endgültige Implementierung angestrebt, sondern eine funktionsfähige prototypische Implementierung, die im kommenden Semester weiter optimiert und ergänzt wird.

\begin{itemize}
\item Architektur und Komponenten
\item GUI
\item IO-Funktionalität
\item Plugin-System
\item Affine Transformationen
\item Unterteilungsalgorithmen
\item Rendering des Kontrollnetzes
\item Beleuchtungsmodi des Rendering
\end{itemize}


\section{Rückblick}

Da für alle Beteiligten der Umgang mit C++ und insbesondere Computergrafik ein neues Themengebiet darstellt, war ein ausführlicher theoretischer Einstieg in das Thema notwendig. 
Hierbei hat sich die Unterstützung durch Herrn Prof. Dr. Georg Umlauf und Pascal Laube als sehr hilfreich erweisen.
In den Projektbesprechungen wurden technische Details geklärt und der theoretische Hintergrund erläutert.
Erst durch Verstehen der Details bezüglich der Unterteilungsalgorithmen, des Renderings und der Anwendbarkeit der Unterteilungsalgorithmen auf verschiedene Kontrollnetze konnten die Tools und Bibliotheken entsprechend evaluiert werden.

Während des Semesters wurden lediglich für sehr kurze Zeiträume und eng eingegrenzte Themengebiete die Aufgaben zwischen den Teammitgliedern verteilt.
Dadurch wurde sichergestellt, dass eine gemeinsame Wissensbasis entsteht, um späteren Missverständnissen vorzubeugen.
Dies hat sich als hilfreich erwiesen und ermöglicht es nun an verschiedenen Teilmodulen der Anwendung parallel zu arbeiten.

