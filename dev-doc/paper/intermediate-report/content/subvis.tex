\chapter{SubVis}

Nachfolgend wird das Programm \emph{SubVis} spezifiziert und dessen Architektur vorgestellt.
Außerdem werden die Entwicklungsorganisation, Dokumentation sowie verwendete Bibliotheken dargelegt.

\section{Anforderungen}

\begin{itemize}
 \item Architektur
 \begin{itemize}
 	\item Erweiterbarkeit durch andere Algorithmen mittels Plugins.
 	\item Plugins GUI TAb zeichnen..
 \end{itemize}
 \item GUI
  \begin{itemize}
 	\item Darstellung des Kontrollnetzes
 	\item Rendern der Fläche
 	\item Rotation des Objektes 
 	\item Translation des Objektes
 	\item Skalierung des Objektes
 	\item Darstellung einer Textur auf der Oberfläche
 	\item Beleuchtung auf der Oberfläche des Netzes, um Glattheit bewerten zu können.
 	\item Edit-Modus: Verschieben eines Punktes anhand seiner Flächennormalen.
 \end{itemize}
 \item Dateiformate/IO
 \begin{itemize}
 	\item OFF-Format und NOFF (mit Farben/Normalen)
 	\item Laden und Speichern
 \end{itemize}
 \item Unterteilungsalgorithmen
 \begin{itemize}
 	\item Catmull-Clark
 	\item Loop
 	\item Doo-Sabin
 	\item Butterfly
 \end{itemize}
 \item Funktionen
 \begin{itemize}
  \item Variable Anzahl von Unterteilungsschritten
  \item Beleuchtungsmodus
  \item eventuell Farbe
 \end{itemize}
\end{itemize}

\section{Bibliotheken und Datenstrukturen}

// TODO , QT, LibQGLViewer, Surface Mesh, OpenGL

\section{Architektur}

// TODO

\section{Dokumentation}

// TODO

\section{Entwicklungsumgebung}

// TODO
C++, Qt Creator, Ubuntu, GIT
