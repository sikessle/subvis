\chapter{Projektverlauf}


\section{Bisheriger Ergebnisse}

Im ersten Semester lag der Schwerpunkt darin, einen Überblick über die theoretischen Grundlagen
zu erhalten. Hierzu gehören die Unterteilungsalgorithmen und die Darstellung der Kontrollnetze.
Ein weiter wichtiger Punkt war die Evaluierung von Bibliotheken, die bereits Algorithmen und
vor allem Datenstrukturen implementieren. Als Ergebnis haben wir die Datenstruktur
Surface Mesh als am geeignetsten bewertet.

\section{Planung für kommendes Semester}

Im kommenden Semester liegt der Fokus in der Implementierung und Umsetzung der Aufgabe.
Das Ergbnis besteht aus dem Programm, dass die Unterteilungsalgorithmen implementiert
und einer schriflichen Ausarbeitung, die den Aufbau und die Umsetzung beschreibt und begründet.

\subsection*{Aufgabenverteilung}
Für die Entwicklung werden die Themen und Aufgaben unter den Teammitgliedern aufgeteilt.
Wir haben für die Entwicklung drei Arbeitspakete definiert.

\begin{description}
\item[grafische Benutzeroberfläche] - Hierzu gehört die Oberfläche, Bedienung und das
Erscheinungsbild des Programms.
\item[Rendering und Darstellung] - In diesem Arbeitspaket soll die Berechnung und Darstellung der
Limesfläche des Kontrollnetzes implementiert werden.
\item[Unterteilungsalgorithmen] - Hierzu gehört die implementierung der vorgegebenen
Unterteilungsalgorithmen.
\end{description}

\autoref{tab:aufgabenverteilung} zeigt die geplante Aufgebeneinteilung für kommendes Semester.
Sollte sich ein Implementierungsteil im Projektverlauf später doch als schwieriger oder
leichter rausstellen, werden dementsprechend die Arbeitspakete angepasst. 

\begin{table}
\caption{Aufgabenverteilung unter den Teammitgliedern}
\center
\begin{tabular}{c|c}
Arbeitspaket & Zugewiesene Personen\\
\hline
grafische Benutzeroberfläche & Simon Kessler \\
Rendering und Darstellung & Tobias Keh \\
Unterteilungsalgorithmen & Felix Born \\
\end{tabular}
\label{tab:aufgabenverteilung}
\end{table}



\section{Rückblick}

// TODO was war gut/schlecht im Team/Zusammenarbeit usw...
