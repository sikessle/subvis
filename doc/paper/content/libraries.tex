\chapter{Bibliotheken}

// TODO Übersicht



\section{OpenMesh und OpenFlipper}

OpenMesh wird von der \acs{RWTH} Aachen entwickelt und steht unter der \acs{LGPL} v3 Lizenz ("with exception"). OpenMesh stellt eine effizente Datenstruktur für polygonale Meshes bereit.

Darauf aufbauend arbeitet das flexible pluginbasierte Framework OpenFlipper, mit dem geometrische Objekte modelliert und verarbeitet werden können. Intern verwendet es OpenMesh. Für die grafische Oberfläche wird QT verwendet.

\subsection{Datenstrukturen und Algorithmen}

OpenMesh implementiert eine Datenstruktur für polygonale Meshes. Es sind sogar bereits Unterteilungsalgorithmen implementiert, die auf dieser Datenstruktur arbeiten.

\begin{enumerate}
\item Uniform subdivision
\begin{itemize}
	\item Loop
	\item SQRT3
	\item Modified Butterfly
	\item Interpolationg SQRT3 LG
	\item Composite
	\item Catmull Clark
\end{itemize}
\item Adaptive subdivision
\begin{itemize}
	\item Adaptive Composite
\end{itemize}
\item Simple subdivision
\begin{itemize}
	\item Longest Edge
\end{itemize}
\end{enumerate}

unterstützte Datenformate: off, obj, ply

// TODO


\section{Surface Mesh}

// TODO


\section{OpenSubdiv}

Bibliothek für schnelles (<3ms) interaktives Rendering von Gitternetzen. Dabei wird mittel paralleler GPU-Berechnung gearbeitet und die Daten intern zwecks Optimierung umgewandelt. Leider wird momentan nur Catmull-Clark unterstützt. Für unsere Zwecke eher nicht geeignet.
// TODO


\section{CGoGN}

// TODO


\section{\acf{CGAL}}

// TODO
